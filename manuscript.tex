\documentclass[a4paper,11pt]{article}
\usepackage[style=mla,style=authoryear,backend=bibtex]{biblatex}
\renewcommand*{\nameyeardelim}{\addcomma\space}  % add comma between author and year
\setcounter{tocdepth}{2}  % Exclude subsubsections from table of content

\addbibresource{bibliography.bib}

\title{Logeti, Stockhausen and Koenig:\\Esthetics of Electronic Music}
\author{Tom Gurion}

\begin{document}

\maketitle
\tableofcontents

\section{Introduction}
\label{sec:introduction}

% Citations tests
Ligeti in conversation: (\cite{varnai}).
Stockhausen paper: (\cite{stockhausen}).
Koenig paper: (\cite{koenig})
Ligeti music: (\cite{rami_music}).
Stockhausen music: (\cite{gesang_music}).
Koenig music: (\cite{todo_music}).
Ligeti score: (\cite{rami}).
Stockhausen score: (\cite{gesang}).
The electronic works of Gy{\"o}rgy Ligeti and their influence on his later style: (\cite{levy2006}).

\section{Historical Review}
\label{sec:historical_Review}

From www.koenigproject.nl (add reference if using):

Gottfried Michael Koenig, born in 1926 in Magdeburg, Germany, studied church music in Braunschweig, composition, piano, analysis and acoustics in Detmold, music representation techniques in Cologne and computer technique in Bonn.
He attended the Darmstadt music summer schools for several years, later as a lecturer.
From 1954 to 1964 Koenig worked in the electronic music studio of West German Radio at Cologne, assisting other composers (including Stockhausen, Kagel, Evangelisti, Ligeti, Brün), and producing his own electronic compositions (Klangfiguren, Essay, Terminus 1).
During this period he also wrote orchestral and chamber music (for piano, string quartet, woodwind quintet).
From 1958 he was an assistant in the radio drama department at the Cologne academy of music, where he taught electronic music, composition and analysis from 1962.
In 1964 Koenig moved to the Netherlands.
Until 1986 he was director and later chairman of the Institute of Sonology at the University of Utrecht.
During this period the Institute acquired a worldwide reputation, particularly for its annual Sonology course.
Koenig also lectured extensively in the Netherlands and other countries and developed his computer programs "Project 1", "Project 2" and "SSP", designed to formalise the composition of musical structure-variants.
He continued to produce electronic works (Terminus 2, the Funktionen series).
These were followed by the application of his computer programs, resulting in chamber music (Übung for piano, the Segmente series, 3 ASKO Pieces, String Quartet 1987, String Trio) and works for orchestra (Beitrag, Concerti e Corali).
Since 1986, when the Institute moved from Utrecht University to the Royal Conservatory at The Hague, Koenig has continued to compose, produce computer graphics and develop musical expert systems.
The first three volumes of his theoretical writings were published between 1991 and 1993 under the title "Ästhetische Praxis" by Pfau Verlag; an Italian selection appeared under the title "Genesi e forma" (Semar, Rome 1995).
A fourth volume followed in 1999, a fifth in 2002; the sixth (2007) contains a complete thematic index.
In 1961 Koenig received an incentive award from the Federal State of North Rhine-Westphalia, in 1987 the Matthijs Vermeulen Prize from the City of Amsterdam, in 1991 the Christoph and Stephan Kaske Prize.
In 2002 the Philosophical Faculty of the University of Saarbrücken, Germany, awarded Koenig an honorary doctorate.
In the winter semester of 2002/2003 he was Visiting Professor for Computer Music at the Technical University, Berlin.
In 2010 Koenig received the Giga-Hertz Prize of ZKM, Karlsruhe.

\section{Research Targets}  % Not sure that this is necessary
\label{sec:research_targets}

\section{Esthetic Ideals in Electronic Music}
\label{sec:esthetic_ideals_in_Electronic_music}

\subsection{Ligeti}
\label{sub:eshtetic_ligeti}

\subsubsection{Hectic quality / Gesticulation / Deep-frozen expressionism}
\label{subs:ligeti:hectic}

I want to remove great, whirling passion, all grand expressive gestures, far away and view them at a distance.
My generation's attitude comes into play here, a rejection of pathos and romanticism (p. 18).

I deprive both pathos and expression of credibility, suddenly everything gets out of gear...
If pathos in a gesture is excessive you can no longer perceive or register it (p. 19).

\subsubsection{Machine like music / Meccanico-type}
\label{subs:ligeti:machine}

I have always been fascinated by machines that do not work properly...
Transposed into music, the ticking of malfunctioning machinery occurs in many of my works (p. 16).

\subsubsection{Micropolyphony}
\label{subs:ligeti:micropolyphony}

Both Atmospheres and Lontano have a dense canonic structure.
But you cannot actually hear the polyphony, the canon.
You hear a kind of impenetrable texture, something like a very densely woven cobweb.
I have retained melodic lines in the process of composition, they are governed by rules as strict as Palestrina’s or those of the Flemish school...
The polyphony structure does not come through, you cannot hear it; it remains hidden in a microscopic, under-water world, to us inaudible (p. 14-15).

\subsubsection{Flexible rhythm}
\label{subs:ligeti:flexible}

What really interested me was flexible rhythm, giving up the pulsating metrical idiom (p. 36).

I built the rhythmic shifts into the music.
For instance, twenty-four violins would play the same tune but with a slight time-lag between them...
No violinist can play them, all of them make mistakes, different mistakes all the time.
These mistakes add up and create a floating, fluctuating patterns (p. 40).

At the time I was interested in the idea of multilayered structures, polyrhythm, and beyond that different tempi, different speed all simultaneously executed (p. 62).

\subsubsection{Overwriting / Misleading notation}
\label{subs:ligeti:overwriting}

My reason for so ‘overwriting’ the score was to achieve the effect I wanted, a sense of danger (p.53).

In a later piece, Ramifications, I divided the strings into two groups with quarter-tone difference in the intonation between them.
This did not produce music based on quarter-tones; that was not my intention...
The point is that as the two groups of strings, deliberately tuned apart from one another, go on playing the group tuned higher automatically slides downwards so that the two groups get nearer one another in pitch.
That is what I wanted: not music based on quarter-tones but mistuned music (p. 54).

...There will be hardly any difference between various performances; the smudginess both in intonation and in rhythm gives the same result, the same degree of ‘dirtiness’ (p. 55).

...The passages written in small notes simply indicates that the player should execute them as fast as he can.
That is why I said that the instrumentalist were not given much freedom; what I kept under control was the overall sound texture at any one time (p. 63).

\subsubsection{``Mistiness'' to ``clearing-up'' process}
\label{subs:ligeti:mistiness}

My idea was that instead of tension-resolution, dissonance-consonance, dominant-tonic, pairs of opposition in traditional tonal music, I would contrast ``mistiness'' with passages of ``clearing-up''. ``Mistiness'' usually means a contrapuntal texture, a micropolyphonic cobweb technique; the perfect interval appears in the texture first as a hint and then gradually becomes the dominant feature (p. 60).

\subsection{Stockhausen}
\label{sub:eshtetic_stockhausen}

On several previous occasions, when I have been asked to explain the composition of electronic music, I have described four characteristics that seem important to me for electronic composition as distinguished from the composition of instrumental music:

\subsubsection{The correlation of the coloristic, harmonic-melodic, and metric-rhythmic aspects of composition}
\label{subs:stockhausen:time}

Most of the current paper by Stockhausen deals with this element.
In my opinion, Stockhausen attitude toward the correlation between different compositional aspects could be put by the two following ideas:

\begin{itemize}
    \item He classify different sonic elements as different perceptions of the time domain.
    For example, periodic elements of short time intervals are intercepted as pitch, and therefore as the basis for melody and harmony;
    Periodic elements of medium time interval are intercepted as tempo and rhythm;
    Those of larger time intervals are seen as musical structure.
    Furthermore, Stockhausen describe those ``time spans'' as spread across 7 octaves:
    Pitch is spread across the 7 octaves of the piano;
    Tempo and rhythm are spread from 1/16 of a second up to 8 seconds;
    Lastly, time spans of 8 to 1024 seconds are intercepted as structural building blocks.

    \item Based on the attitude described above Stockhausen approach electronic music as a tool to apply total serializm (TODO: {total serializm reference}).
    Furthermore, he argues that serializm must exist in electronic composition.
    The thinking path the led Stockhausen to this argument is clear, but it is still hard for me to agree with him.
    Later, I will try to confront this idea with those reflected in his and Ligeti's music.
\end{itemize}

Overall, I can not find a common ground between this compository element to those of Ligeti that were mentioned before.
Their attitudes regarding unified perception of time are, most probably, contradicting.

\subsubsection{The composition and de-composition of timbres}
\label{subs:stockhausen:timber}

Ligeti reference this ideas as following:
Modification of timbre and dynamics are obviously very significant but the patterns emerging from them are even more important...
I should say that the changes and modifications of the overall pattern are the important feature, not the tone colors. (Ligeti in conversation, p. 39).

\subsubsection{The characteristic differentiation among degrees of intensity}
\label{subs:stockhausen:intensity}

I'm not sure I fully understand the meaning of this element, not from the current paper, at least.
I will leave this element ``out of scope'' for now.

\subsubsection{The ordered relationships between sound and noise}
\label{subs:stockhausen:noise}

main area to compare Ligeti and Stockhausen ideas TODO.

\subsection{Koenig}
\label{sub:eshtetic_koenig}

\section{Compositions}
\label{sec:compositions}

\subsection{Ligeti: Remifications}
\label{sub:composition_ligeti}

Ligeti ideals and characteristics as found in Remifications.

\subsubsection{Micropolyphony, flexible rhythm and misleading notation}

This ideas are perhaps the most dominants in this composition.
Examples can be found in the 0:00 - 1:55, 4:20 - 4:35, 5:28 - 6:25.

\subsubsection{Machine like music / Meccanico-type}

It appears to me that these idea could be counted together with the previous ones.
On the other hand, some parts in Remifications are dominated by more ``machine-like'' characteristics than others.
For example: 5:28 - 6:25, 7:00 - 8:00.

\subsubsection{Hectic quality / Gesticulation / Deep-frozen expressionism}

Gesticulated phrases appears in Remifications only for short periods, they are not characterize the composition as a hole.
Nevertheless, we can note two ``overly exaggerated'' phrases, where the primer is a bit more distinct in that manner: 3:36 - 3:45, 6:28 - 6:40.
One could argue that these are just dramatic phrases within the composition, but I suspect that this amount of vibrato in Ligeti's composition most present some decent amount of cynicism.

\subsubsection{‘Mistiness’ to ‘clearing-up’ process}

In my opinion this idea is the lessen distinct in this composition.
This is probably because of the firm voice that accompanies any of the mictrotonal parts, while the ``mistiness'' microtonal is steady and is not ``clearing-up'' into an perfect interval.
The most relevant parts I managed to find for this idea are: 1:40 - 2:08, 2:19 - 2:38.

\subsection{Stockhausen: Gesang der J{\"u}nglinge}
\label{sub:composition_stockhausen}

\subsection{Koenig: TODO}
\label{sub:composition_koenig}

\section{General Discussion}
\label{sec:general_discussion}

\printbibliography[title={Bibliography}]

\end{document}
