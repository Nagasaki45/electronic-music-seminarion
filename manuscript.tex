\documentclass[a4paper,11pt]{article}
\usepackage[style=mla,style=authoryear,backend=bibtex]{biblatex}
\renewcommand*{\nameyeardelim}{\addcomma\space}  % add comma between author and year
\setcounter{tocdepth}{2}  % Exclude subsubsections from table of content

% -------- Quoting ------------
\usepackage{mdframed}
\usepackage{xcolor}

\mdfdefinestyle{MyShadeQuoteStyle}{%
    leftmargin=15pt,
    rightmargin=15pt,
    backgroundcolor=gray!25,
    linewidth=0pt,
    skipbelow=\topskip,
    skipabove=\topskip
}

\newenvironment{MyShadequote}[1][]{%
    \ignorespaces%
    \begin{mdframed}[style=MyShadeQuoteStyle,#1]%
}{%
    \end{mdframed}%
    \ignorespacesafterend%
}%
% -------- \Quoting ------------

\addbibresource{bibliography.bib}

\title{Ligeti, Stockhausen and Koenig:\\Aesthetics of Electronic Music}
\author{Tom Gurion}

\begin{document}

\maketitle
\tableofcontents

\section{Introduction}
\label{sec:introduction}

% Citations tests
Ligeti in conversation: (\cite{varnai}).
Stockhausen paper: (\cite{stockhausen}).
Koenig paper: (\cite{koenig})
Ligeti music: (\cite{rami_music}).
Stockhausen music: (\cite{gesang_music}).
Koenig music: (\cite{todo_music}).
Ligeti score: (\cite{rami}).
Stockhausen score: (\cite{gesang}).
The electronic works of Gy{\"o}rgy Ligeti and their influence on his later style: (\cite{levy2006}).

\section{Historical Review}
\label{sec:historical_Review}

Gy{\"o}rgy Ligeti, one of the most innovative musicians of modern 20\textsuperscript{th}-century, was born in Hungary to a Jewish family in 1923.
After suffering from two tyrannies in his youth, Nazi and Stalinist, he left Hungary in 1956 facing western Europe (\cite{ligeti_grove}).
There, he was exposed to electronic music through the Darmstadt-Cologne avant guard ideology, which influenced his writing ever since (\cite[p. TODO]{levy2006}).

Karlheinz Stockhausen, probably the most prominent German composer of the 2\textsuperscript{nd} half of the 20\textsuperscript{th}-century, was born in Cologne in 1928.
He enrolled at the Cologne Musikhochschule in 1947 and graduated in 1951.
After his graduation Stockhausen went to the Darmstadt Internationale Ferienkurse f{\"u}r Neue Musik (the International Summer Courses for New Music) and later moved to Paris, where he met Messiaen, Boulez and Pierre Schaeffer and introduced to the Parisian avant guard and the musique-concr{\`e}te studios.

By 1953 Stockhausen was already established as a leading serialist avant guard composer.
Over the next few years he became the leading figure in Darmstadt-Cologne avant guard scene and from 1956 started to teach regularly at the Darmstadt summer courses (\cite{stockhausen_grove}).

Gottfried Michael Koenig, was born in 1926 in Magdeburg, Germany.
He studied church music in Braunschweig, composition, piano, analysis and acoustics in Detmold, music representation techniques in Cologne and computer technique in Bonn\footnote{Koenig homepage: www.koenigproject.nl/indexe.html}.
During the 1951 Darmstadt summer courses he attended lectures by Meyer-Eppler that awakened his interest in electronic sound production.
In 1953 he moved to Cologne where he studied music technology at the Cologne Hochschule f{\"u}r Music and attended courses on electronic data processing at the University of Cologne.
At the same time he started to work at the studios for electronic music in Cologne, the Nordwest Deutscher Radio (NWDR), first as assistant for other composers and later as a permanent employee and composer (\cite{koenig_grove}).

Darmstadt of post Wold War II became an important center for modern music thereby an attractive place for young composers.
The city is located in the state of Hessen in central Germany, in the American zone of occupation of those days.
The summer courses were financial supported by the US military government with two goals:
first, to propagate American values in effort to reeducate the German population in preparation for the establishment of democratic institutions;
and second, to provide a meeting place where musicians from the former fascist or fascist-occupied areas might further their musical reeducation through exposure to styles and techniques that had been prohibited during the fascist years (\cite{darmstadt_oxford}).
Since the first Darmstadt summer course, in 1948, the focus was on current musical concepts;
at the time it was, mainly, electronic and serail music\footnote{Internationales Musicinstitut Darmstadt: www.internationales-musikinstitut.de/en/summer-course/history/210-history.html}.

In addition to Darmstadt summer courses, the electronic music studios of Cologne were another prominent center for the Darmstadt-Cologne avant guard scene.
The studios were founded in 1951, focused on the creation of pure electronic or ``electroacoustic'' music.
It is believed that the term ``electroacoustic'' was coined by Meyer-Eppler, one of the founders of the studios, to distinguish themselves from musique-concr{\`e}te:
a technique that characterized the already established style of Parisian electronic music (\cite{nwdr}).

The three composers that the current work is about --- Ligeti, Stockhausen and Koenig --- found themselves, each one in its own way, taking part in the Darmstadt-Cologne scene during the 1950\textsuperscript{s}.
Moreover, they all were in a similar stage of their career when they exposed to the ideas presented in the summer courses and the techniques used in the electronic music studios.
These ideas deeply influenced their styles as composers (TODO cite).
However, whereas Stockhausen and Koenig music is a direct continuation of these ideas, Ligeti music cannot be perceived as part of Darmstadt-Cologne style (TODO cite);
his main repertoire is acoustic, and his works are usually not categorized as serial music.

\section{Research Targets}  % Not sure that this is necessary
\label{sec:research_targets}

\section{Aesthetic Ideals in Electronic Music}
\label{sec:aesthetic_ideals_in_electronic_music}

In this section I analyze musical concepts as described in \textbf{written materials} by the three composers.
For each composer I analyze one written source:
for Ligeti, I analyze ``Ligeti in conversation'' --- an interview that was conducted by Peter Varnai in 1983 (\cite{varnai});
for Stockhausen, I analyze his 1962 paper ``The concept of unity in electronic music'' (\cite{stockhausen});
and for Koenig, I analyze his 1963 paper ``The Construction of Sound'' (\cite{koenig}).

This work deals with aesthetics of electronic music, hence the decision to choose these papers by Stockhausen and Koenig is trivial.
On the other hand, for Ligeti I choose a source with broader perspective, as his music is usually not categorized as electronic.
Varnai's interview of Ligeti presents a comprehensive set of ideas behind Ligeti's composition style, and I found it useful as a basis to compare the musical concepts with.

Later in this section I will compare these ideas, and search for similarities and differences between them.
A unified understanding of the different ideas by the different composers will be used to analyze their music in the next section.

\subsection{Ligeti}
\label{sub:eshtetic_ligeti}

The following is a curated list musical concepts that I have collected from Ligeti's interview.
Ligeti was not list his concepts explicitly, hence the categorization is done by me and is also subjective.
The description of each concept is quoted directly from the interview.
I add additional clarification only when absolutely necessary, as I think that the concepts are relatively self-explanatory.

\subsubsection{Hectic quality / Gesticulation / Deep-frozen expressionism}
\label{subs:ligeti:hectic}

\begin{MyShadequote}
  I want to remove great, whirling passion, all grand expressive gestures, far away and view them at a distance.
  My generation's attitude comes into play here, a rejection of pathos and romanticism (p. 18).
\end{MyShadequote}

\begin{MyShadequote}
  I deprive both pathos and expression of credibility, suddenly everything gets out of gear...
  If pathos in a gesture is excessive you can no longer perceive or register it (p. 19).
\end{MyShadequote}

Indeed, researchers often describe Ligeti's music as pale and nervous dream-like (or even nightmarish) reflection of life.
``Grand emotions are captured in a manner that they do not sound real or believable'' argues Frigyesi, ``gestures exaggerate to the point that it becomes surreal, frightening, absurd, or even comical'' (TODO Frigeyesi).

\subsubsection{Machine like music / Meccanico-type}
\label{subs:ligeti:machine}

\begin{MyShadequote}
  I have always been fascinated by machines that do not work properly...
  Transposed into music, the ticking of malfunctioning machinery occurs in many of my works (p. 16).
\end{MyShadequote}

In addition, researchers pointed out that Ligeti's machines always brake down in his music, what makes them only more frightening (TODO cite).

\subsubsection{Micropolyphony}
\label{subs:ligeti:micropolyphony}

\begin{MyShadequote}
  Both Atmospheres and Lontano have a dense canonic structure.
  But you cannot actually hear the polyphony, the canon.
  You hear a kind of impenetrable texture, something like a very densely woven cobweb.
  I have retained melodic lines in the process of composition, they are governed by rules as strict as Palestrina’s or those of the Flemish school...
  The polyphony structure does not come through, you cannot hear it; it remains hidden in a microscopic, under-water world, to us inaudible (p. 14-15).
\end{MyShadequote}

\subsubsection{Flexible rhythm}
\label{subs:ligeti:flexible}

\begin{MyShadequote}
  What really interested me was flexible rhythm, giving up the pulsating metrical idiom (p. 36).
\end{MyShadequote}

\begin{MyShadequote}
  I built the rhythmic shifts into the music.
  For instance, twenty-four violins would play the same tune but with a slight time-lag between them...
  No violinist can play them, all of them make mistakes, different mistakes all the time.
  These mistakes add up and create a floating, fluctuating patterns (p. 40).
\end{MyShadequote}

\begin{MyShadequote}
  At the time I was interested in the idea of multilayered structures, polyrhythm, and beyond that different tempi, different speed all simultaneously executed (p. 62).
\end{MyShadequote}

\subsubsection{Overwriting / Misleading notation}
\label{subs:ligeti:overwriting}

\begin{MyShadequote}
  My reason for so `overwriting' the score was to achieve the effect I wanted, a sense of danger (p.53).
\end{MyShadequote}

\begin{MyShadequote}
  In a later piece, Ramifications, I divided the strings into two groups with quarter-tone difference in the intonation between them.
  This did not produce music based on quarter-tones; that was not my intention...
  The point is that as the two groups of strings, deliberately tuned apart from one another, go on playing the group tuned higher automatically slides downwards so that the two groups get nearer one another in pitch.
  That is what I wanted: not music based on quarter-tones but mistuned music (p. 54).
\end{MyShadequote}

\begin{MyShadequote}
  ...There will be hardly any difference between various performances; the smudginess both in intonation and in rhythm gives the same result, the same degree of ‘dirtiness’ (p. 55).
\end{MyShadequote}

\begin{MyShadequote}
  ...The passages written in small notes simply indicates that the player should execute them as fast as he can.
  That is why I said that the instrumentalist were not given much freedom; what I kept under control was the overall sound texture at any one time (p. 63).
\end{MyShadequote}

See also the 2\textsuperscript{nd} quote from last subsection (Flexible rhythm -- \ref{subs:ligeti:flexible}).

\subsubsection{``Mistiness'' to ``clearing-up'' process}
\label{subs:ligeti:mistiness}

\begin{MyShadequote}
  My idea was that instead of tension-resolution, dissonance-consonance, dominant-tonic, pairs of opposition in traditional tonal music, I would contrast ``mistiness'' with passages of ``clearing-up''. ``Mistiness'' usually means a contrapuntal texture, a micropolyphonic cobweb technique; the perfect interval appears in the texture first as a hint and then gradually becomes the dominant feature (p. 60).
\end{MyShadequote}

\subsection{Stockhausen}
\label{sub:eshtetic_stockhausen}

Compared to Ligeti, Stockhausen explicitly count 4 characteristics of electronic compositions.
His paper, ``The Concept of Unity in Electronic Music'' opens as following.

\begin{MyShadequote}
  On several previous occasions, when I have been asked to explain the composition of electronic music, I have described four characteristics that seem important to me for electronic composition as distinguished from the composition of instrumental music:
\end{MyShadequote}

The titles of the next subsections, \ref{subs:stockhausen:time} -- \ref{subs:stockhausen:noise}, were copied directly from Stockhausen's paper.
In each subsection I add clarification, based on the rest of the paper, as needed.

\subsubsection{The correlation of the coloristic, harmonic-melodic, and metric-rhythmic aspects of composition}
\label{subs:stockhausen:time}

Most of the current paper by Stockhausen deals with this element.
In my opinion, Stockhausen attitude toward the correlation between different compositional aspects could be described by two ideas.

\begin{itemize}
    \item He classifies different sonic elements as different perceptions of the time domain.
    For example, periodic elements of short time intervals are intercepted as pitch, and therefore as the basis for melody and harmony;
    Periodic elements of medium time interval are intercepted as tempo and rhythm;
    Those of larger time intervals supports the musical structure.
    Furthermore, according to Stockhausen these ``time spans'' range across 7 octaves:
    Pitch spread across the 7 octaves of the piano;
    Tempo and rhythm range from 1/16 of a second up to 8 seconds;
    Lastly, time spans of 8 to 1024 seconds are intercepted as structural building blocks.

    \item Based on the above, Stockhausen approaches to electronic music as a tool for total serializm (TODO: {total serializm reference}).
    Furthermore, he argues that serializm must exist in electronic composition, inherently.
    The thinking path that led Stockhausen to this argument is clear, but it is still hard for me to agree with him.
    Later, I will try to confront this idea with those reflected in his and Ligeti's music.
\end{itemize}

\subsubsection{The composition and de-composition of timbres}
\label{subs:stockhausen:timber}

Ligeti also refer to this ideas:

\begin{MyShadequote}
  Modification of timbre and dynamics are obviously very significant but the patterns emerging from them are even more important...
  I should say that the changes and modifications of the overall pattern are the important feature, not the tone colors. (Ligeti in conversation, p. 39).
\end{MyShadequote}

\subsubsection{The characteristic differentiation among degrees of intensity}
\label{subs:stockhausen:intensity}

I'm not sure I fully understand the meaning of this element, not from the current paper, at least.
I will leave this element ``out of scope'' for now.

\subsubsection{The ordered relationships between sound and noise}
\label{subs:stockhausen:noise}

main area to compare Ligeti and Stockhausen ideas TODO.

\subsection{Koenig}
\label{sub:eshtetic_koenig}

\subsection{Comparison}

TODO regarding the first concept by Stockhausen:
Overall, I can not find a common ground between this compository element to those of Ligeti that were mentioned before.
Their attitudes regarding unified perception of time are, most probably, contradicting.

\section{Compositions}
\label{sec:compositions}

\subsection{Ligeti: Remifications}
\label{sub:composition_ligeti}

Ligeti ideals and characteristics as found in Remifications.

\subsubsection{Micropolyphony, flexible rhythm and misleading notation}

This ideas are perhaps the most dominants in this composition.
Examples can be found in the 0:00 - 1:55, 4:20 - 4:35, 5:28 - 6:25.

\subsubsection{Machine like music / Meccanico-type}

It appears to me that these idea could be counted together with the previous ones.
On the other hand, some parts in Remifications are dominated by more ``machine-like'' characteristics than others.
For example: 5:28 - 6:25, 7:00 - 8:00.

\subsubsection{Hectic quality / Gesticulation / Deep-frozen expressionism}

Gesticulated phrases appears in Remifications only for short periods, they are not characterize the composition as a hole.
Nevertheless, we can note two ``overly exaggerated'' phrases, where the primer is a bit more distinct in that manner: 3:36 - 3:45, 6:28 - 6:40.
One could argue that these are just dramatic phrases within the composition, but I suspect that this amount of vibrato in Ligeti's composition most present some decent amount of cynicism.
TODO cite frigyesi to support this statement.

\subsubsection{‘Mistiness’ to ‘clearing-up’ process}

In my opinion this idea is the lessen distinct in this composition.
This is probably because of the firm voice that accompanies any of the mictrotonal parts, while the ``mistiness'' microtonal is steady and is not ``clearing-up'' into an perfect interval.
The most relevant parts I managed to find for this idea are: 1:40 - 2:08, 2:19 - 2:38.

\subsection{Stockhausen: Gesang der J{\"u}nglinge}
\label{sub:composition_stockhausen}

\subsection{Koenig: TODO}
\label{sub:composition_koenig}

\section{General Discussion}
\label{sec:general_discussion}

\printbibliography[title={Bibliography}]

\end{document}
