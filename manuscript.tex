\documentclass[a4paper,11pt]{article}
\usepackage[style=mla,style=authoryear,backend=bibtex]{biblatex}
\renewcommand*{\nameyeardelim}{\addcomma\space}  % add comma between author and year
\setcounter{tocdepth}{2}  % Exclude subsubsections from table of content

\usepackage{graphicx}

% -------- Quoting ------------
\usepackage{mdframed}
\usepackage{xcolor}

\mdfdefinestyle{MyShadeQuoteStyle}{%
    leftmargin=15pt,
    rightmargin=15pt,
    backgroundcolor=gray!25,
    linewidth=0pt,
    skipbelow=\topskip,
    skipabove=\topskip
}

\newenvironment{MyShadequote}[1][]{%
    \ignorespaces%
    \begin{mdframed}[style=MyShadeQuoteStyle,#1]%
}{%
    \end{mdframed}%
    \ignorespacesafterend%
}%
% -------- \Quoting ------------

\addbibresource{bibliography.bib}

\title{Ligeti, Stockhausen and Koenig:\\Aesthetics of Electronic Music}
\author{Tom Gurion}

\begin{document}

\maketitle
\tableofcontents

\section{Introduction and targets}
\label{sec:introduction}

Electronic music is usually defined as music that produced using electronic devices (TODO cite).
In addition to creation techniques, electronic music presents several unique aesthetic characteristics.
These where explored by researchers and composers since the early forays into the field in the late 1940\textsubscript{s}.

The current work deals with the following research question:
\emph{Are aesthetics of electronic music obligating the use of electronic sound creation?}

In order to answer this question I will collect ideas and concepts by three of the greatest composers of the 2\textsuperscript{nd} half of the 20\textsuperscript{th} century: Gy{\"o}rgy Ligeti, Karlheinz Stockhausen, and Gottfried Michael Koenig.
Some of these ideas, namely, those of Stockhausen and Koenig, represents aesthetic values of electronic music explicitly.
Ligeti, on the other hand, is not an electronic music composer;
the concepts that he presents encapsulate his attitude towards music and composition in general.

After listing and explaining the different ideas I will try to find common concepts.
Throughout this process I hypothesize that most electronic music aesthetics can be realized in Ligeti's instrumental music.

Finally I will analyze a piece by Ligeti --- Ramification, for 12 solo strings (\cite{rami}) --- to support my argument.
I will show that most of the different concepts, those of Ligeti, that explicitly refer to instrumental music, as well as those of Stockhausen and Koenig, which deals with electronic music explicitly, can be used to understand this instrumental piece.

More generally, understanding electronic music as \textbf{a way of thinking} rather than a way to create sound using electronic devices may shed a light over instrumental music by composers with former electronic music education as Ligeti, and even more important, on instrumental music that lives side by side with electronically created music for almost 70 years.

\section{Historical Review}
\label{sec:historical_Review}

Gy{\"o}rgy Ligeti, one of the most innovative musicians of modern 20\textsuperscript{th}-century, was born in Hungary to a Jewish family in 1923.
After suffering from two tyrannies in his youth, Nazi and Stalinist, he left Hungary in 1956 facing western Europe (\cite{ligeti_grove}).
There, he was exposed to electronic music through the Darmstadt-Cologne avant guard ideology, which influenced his writing ever since (\cite[p. TODO]{levy2006}).

Karlheinz Stockhausen, probably the most prominent German composer of the 2\textsuperscript{nd} half of the 20\textsuperscript{th}-century, was born in Cologne in 1928.
He enrolled at the Cologne Musikhochschule in 1947 and graduated in 1951.
After his graduation Stockhausen went to the Darmstadt Internationale Ferienkurse f{\"u}r Neue Musik (the International Summer Courses for New Music) and later moved to Paris, where he met Messiaen, Boulez and Pierre Schaeffer and introduced to the Parisian avant guard and the musique-concr{\`e}te studios.

By 1953 Stockhausen was already established as a leading serialist avant guard composer.
Over the next few years he became the leading figure in Darmstadt-Cologne avant guard scene and from 1956 started to teach regularly at the Darmstadt summer courses (\cite{stockhausen_grove}).

Gottfried Michael Koenig, was born in 1926 in Magdeburg, Germany.
He studied church music in Braunschweig, composition, piano, analysis and acoustics in Detmold, music representation techniques in Cologne and computer technique in Bonn\footnote{Koenig homepage: www.koenigproject.nl/indexe.html}.
During the 1951 Darmstadt summer courses he attended lectures by Meyer-Eppler that awakened his interest in electronic sound production.
In 1953 he moved to Cologne where he studied music technology at the Cologne Hochschule f{\"u}r Music and attended courses on electronic data processing at the University of Cologne.
At the same time he started to work at the studios for electronic music in Cologne, the Nordwest Deutscher Radio (NWDR), first as assistant for other composers and later as a permanent employee and composer (\cite{koenig_grove}).

Darmstadt of post Wold War II became an important center for modern music and thereby an attractive place for young composers.
The city is located in the state of Hessen in central Germany, in the American zone of occupation of those days.
The summer courses were financial supported by the US military government with two goals:
first, to propagate American values in effort to reeducate the German population in preparation for the establishment of democratic institutions;
and second, to provide a meeting place where musicians from the former fascist or fascist-occupied areas might further their musical reeducation through exposure to styles and techniques that had been prohibited during the fascist years (\cite{darmstadt_oxford}).
Since the first Darmstadt summer course, in 1948, the focus was on current musical concepts;
at the time it was, mainly, electronic and serail music\footnote{Internationales Musicinstitut Darmstadt: www.internationales-musikinstitut.de/en/summer-course/history/210-history.html}.

In addition to Darmstadt summer courses, the electronic music studios of Cologne --- the NWDR --- were another prominent center for the Darmstadt-Cologne avant guard scene.
The studios were founded in 1951, focused on the creation of pure electronic or ``electroacoustic'' music.
It is believed that the term ``electroacoustic'' was coined by Meyer-Eppler, one of the founders of the studios, to distinguish themselves from musique-concr{\`e}te:
a technique that characterized the already established style of Parisian electronic music\footnote{Martian Arts -- The Birth of Electronic Music Recording Studios: http://www.martianarts.net/web/culture/technology/105-the-birth-of-electronic-music-recording-studios}.

The three composers that the current work is about --- Ligeti, Stockhausen and Koenig --- found themselves, each one in its own way, taking part in the Darmstadt-Cologne scene during the 1950\textsuperscript{s}.
Moreover, they all were in a similar stage of their career when they exposed to the ideas presented in the summer courses and the techniques used in the electronic music studios.
These ideas deeply influenced their styles as composers (TODO cite).
However, whereas Stockhausen and Koenig musical styles are direct continuations of these ideas, Ligeti's music cannot be perceived as part of Darmstadt-Cologne style (TODO cite);
his main repertoire is acoustic, and his works are usually not categorized as serial music.

\section{Aesthetic Ideals in Electronic Music}
\label{sec:aesthetic_ideals_in_electronic_music}

In this section I analyze musical concepts as described in \textbf{written materials} by the three composers.
For each composer I analyze one written source:
for Ligeti, I analyze ``Ligeti in conversation'' --- an interview that was conducted by Peter Varnai in 1983 (\citeauthor{varnai});
for Stockhausen, I analyze his 1962 paper ``The concept of unity in electronic music'' (\citeauthor{stockhausen});
and for Koenig, I analyze his 1963 paper ``The Construction of Sound'' (\citeauthor{koenig}).

This work deals with aesthetics of electronic music, hence the decision to choose these papers by Stockhausen and Koenig is trivial.
On the other hand, for Ligeti I choose a source with broader perspective, as his music is usually not categorized as electronic.
Varnai's interview of Ligeti presents a comprehensive set of ideas behind Ligeti's musical style, and I found it useful as a basis to compare the musical concepts with.

Some of the following concepts are my summaries to the ideas found in the written materials, whereas others are quoted directly from the source.
For them, I add clarification only when absolutely necessary, as I think that most of the concepts are relatively self-explanatory.

Later in this section I will compare these ideas, and search for similarities and differences between them.
A unified understanding of the different ideas by the different composers will be used in the next section to analyze Ligeti's Ramifications.

\subsection{Ligeti}
\label{sub:eshtetic_ligeti}

The following is a curated list of musical concepts that I have collected from Ligeti's interview.
Ligeti was not listed his concepts explicitly, hence the categorization is done by me and is also subjective.

\subsubsection{Hectic quality / Gesticulation / Deep-frozen expressionism}
\label{subs:ligeti:hectic}

\begin{MyShadequote}
  I want to remove great, whirling passion, all grand expressive gestures, far away and view them at a distance.
  My generation's attitude comes into play here, a rejection of pathos and romanticism (p. 18).
\end{MyShadequote}

\begin{MyShadequote}
  I deprive both pathos and expression of credibility, suddenly everything gets out of gear...
  If pathos in a gesture is excessive you can no longer perceive or register it (p. 19).
\end{MyShadequote}

Indeed, researchers often describe Ligeti's music as pale and nervous dream-like (or even nightmarish) reflection of life.
``Grand emotions are captured in a manner that they do not sound real or believable'' argues Frigyesi, ``gestures exaggerate to the point that it becomes surreal, frightening, absurd, or even comical'' (TODO Frigeyesi).

\subsubsection{Machine like music / Meccanico-type}
\label{subs:ligeti:machine}

\begin{MyShadequote}
  I have always been fascinated by machines that do not work properly...
  Transposed into music, the ticking of malfunctioning machinery occurs in many of my works (p. 16).
\end{MyShadequote}

In addition, researchers pointed out that Ligeti's machines always brake down in his music, what makes them only more frightening (TODO cite).

\subsubsection{Micropolyphony}
\label{subs:ligeti:micropolyphony}

\begin{MyShadequote}
  Both Atmospheres and Lontano have a dense canonic structure.
  But you cannot actually hear the polyphony, the canon.
  You hear a kind of impenetrable texture, something like a very densely woven cobweb.
  I have retained melodic lines in the process of composition, they are governed by rules as strict as Palestrina’s or those of the Flemish school...
  The polyphony structure does not come through, you cannot hear it; it remains hidden in a microscopic, under-water world, to us inaudible (p. 14-15).
\end{MyShadequote}

\subsubsection{Flexible rhythm}
\label{subs:ligeti:flexible}

\begin{MyShadequote}
  What really interested me was flexible rhythm, giving up the pulsating metrical idiom (p. 36).
\end{MyShadequote}

\begin{MyShadequote}
  I built the rhythmic shifts into the music.
  For instance, twenty-four violins would play the same tune but with a slight time-lag between them...
  No violinist can play them, all of them make mistakes, different mistakes all the time.
  These mistakes add up and create a floating, fluctuating patterns (p. 40).
\end{MyShadequote}

\begin{MyShadequote}
  At the time I was interested in the idea of multilayered structures, polyrhythm, and beyond that different tempi, different speed all simultaneously executed (p. 62).
\end{MyShadequote}

\subsubsection{Overwriting / Misleading notation}
\label{subs:ligeti:overwriting}

\begin{MyShadequote}
  My reason for so `overwriting' the score was to achieve the effect I wanted, a sense of danger (p.53).
\end{MyShadequote}

\begin{MyShadequote}
  In a later piece, Ramifications, I divided the strings into two groups with quarter-tone difference in the intonation between them.
  This did not produce music based on quarter-tones; that was not my intention...
  The point is that as the two groups of strings, deliberately tuned apart from one another, go on playing the group tuned higher automatically slides downwards so that the two groups get nearer one another in pitch.
  That is what I wanted: not music based on quarter-tones but mistuned music (p. 54).
\end{MyShadequote}

\begin{MyShadequote}
  ...There will be hardly any difference between various performances; the smudginess both in intonation and in rhythm gives the same result, the same degree of ‘dirtiness’ (p. 55).
\end{MyShadequote}

\begin{MyShadequote}
  ...The passages written in small notes simply indicates that the player should execute them as fast as he can.
  That is why I said that the instrumentalist were not given much freedom; what I kept under control was the overall sound texture at any one time (p. 63).
\end{MyShadequote}

See also the 2\textsuperscript{nd} quote from last subsection (Flexible rhythm -- \ref{subs:ligeti:flexible}).

\subsubsection{``Mistiness'' to ``clearing-up'' process}
\label{subs:ligeti:mistiness}

\begin{MyShadequote}
  My idea was that instead of tension-resolution, dissonance-consonance, dominant-tonic, pairs of opposition in traditional tonal music, I would contrast ``mistiness'' with passages of ``clearing-up''. ``Mistiness'' usually means a contrapuntal texture, a micropolyphonic cobweb technique; the perfect interval appears in the texture first as a hint and then gradually becomes the dominant feature (p. 60).
\end{MyShadequote}

\subsection{Stockhausen}
\label{sub:eshtetic_stockhausen}

Compared to Ligeti, Stockhausen explicitly count 4 characteristics of electronic compositions.
His paper, ``The Concept of Unity in Electronic Music'' opens as following.

\begin{MyShadequote}
  On several previous occasions, when I have been asked to explain the composition of electronic music, I have described four characteristics that seem important to me for electronic composition as distinguished from the composition of instrumental music:

  \begin{itemize}
    \item The correlation of the coloristic, harmonic-melodic, and metric-rhythmic aspects of composition
    \item The composition and de-composition of timbres
    \item The characteristic differentiation among degrees of intensity
    \item The ordered relationships between sound and noise
  \end{itemize}

  Here, I would like to discuss only the correlation of timbre, pitch, intensity, and duration (p. 39).
\end{MyShadequote}

Stockhausens's paper deals mainly with the 1\textsuperscript{st} point above.
However, I will try to glean additional characteristics of electronic music from the paper as well.

\subsubsection{The correlation of timbre, pitch, intensity, and duration}
\label{subs:stockhausen:time}

Stockhausen classifies different sonic elements as different perceptions of the time domain, thus suggesting a unified time model for electronic music.
Periodic elements of short time intervals are intercepted as pitch, and therefore as the basis for melody and harmony;
Periodic elements of medium time interval are intercepted as tempo and rhythm;
Those of larger time intervals supports the musical structure.
Furthermore, according to Stockhausen these ``time spans'' range across 7 octaves each:
Pitch spread across the 7 octaves of the piano;
Tempo and rhythm range from 1/16 of a second up to 8 seconds;
Lastly, time spans of 8 to 1024 seconds are intercepted as structural building blocks.

Based on the above, Stockhausen approaches electronic music as a tool for total serializm (TODO total serializm reference).
Furthermore, he argues that serializm must exist in electronic composition, inherently.

\subsubsection{Lower level building blocks}
\label{subs:stockhausen:sound}

\begin{MyShadequote}
  A musical composition is no more than a temporal ordering of sound events, just as each sound event in a composition is a temporal ordering of pulses.
  It is only a question of the point at which composition begins:
  in composing for instruments whose sounds are predetermined, a composer need not be concerned with these problems.
  On the other hand, in electronic music, one can either compose each sound directly in terms of its wave succession, or, finally, each individual sound wave may be determined in terms of its actual vibration, by an ordering of the succession of pulses (p. 41).
\end{MyShadequote}

\begin{MyShadequote}
  One would not proceed from sound properties that had already been experienced and then allow these to determine temporal variations;
  instead, one would compose the temporal arrangements of pulses themselves, and discover their resultant sound properties experimentally (p. 42).
\end{MyShadequote}

\subsubsection{Noise}
\label{subs:stockhausen:noise}

Stockhausen notes that in traditional instrumental music harmony and melody are always based on whole-number divisions of pitches.
Therefor, ``indicating the necessity of excluding noise from this kind of music'' (\cite[p. 47]{stockhausen}).
This idea shifts naturally to rhythm, and from there, similarly, to form and structure.

Referring back to Stockhausen first concept (in section \ref{subs:stockhausen:time}), a unified model for time in music solves the above by introducing ``noise'' in all different ranges of the time domain -- from timbre to form.

\subsection{Koenig}
\label{sub:eshtetic_koenig}

``The Construction of Sound'' by Koenig (\cite*{koenig}) deals mainly with serial music and in different ways that serial decisions can be incorporated into electronic music.
The current work explores aesthetics in electronic music.
Thus, in this section I try to extract electronic music characteristics from the serial music context in which they are presented in the paper.

\subsubsection{Sounds of unlimited lengths}
\label{subs:koenig:lengths}

\begin{MyShadequote}
  (Intrumental music characterized by) homogeneous, continuous sounds of limited lengths.
  One might say that all classic categories of melodic formation, counterpoint, harmony, form and instrumentation have their origins in this manner of production.
  (On the other hand), the length of electronically produced sound is theoretically unlimited...
  The classic categories thus do not apply to electronic music;
  at any rate they do not have their origins in the new material and the way in which it is produced (p. 1-2).
\end{MyShadequote}

\subsubsection{Non-stationary sounds}
\label{subs:koenig:nonstationary}

Koenig classify timbre into ``classes''.
Both instrumental and electronic music share three classes of continuous sounds:
Harmonic sound, which consists a fundamental; Noise sound, without fundamental; and impulses, which have no timbre nor fundamental.
According to Koenig ``(for the three classes above) we have one timbre for each attack'' (p. 17).

Electronic music enhances the timbre possibilities with another class:

\begin{MyShadequote}
 Electronic music also knows a family of non-stationary sounds with a duration, but no constant timbre...
 Since their ``timbre'' is not constant, it can be described but not accurately imagined.
 The movement of the timbre rather follows the compositional process of the micro-time range.
 In this ways structures are composed instead of timbre (p. 18).
\end{MyShadequote}

\subsubsection{Sound as a formal event}
\label{subs:koenig:event}

\begin{MyShadequote}
  The electronically produced sound does not have to be homogeneous and continuous.
  One complete sound can consist of several individual components whose determinants may change in the next sound.
  The sound is a closed unit and becomes a means of formation;
  each sound characteristic becomes a formal event.
  Instrumental music composes ``with'' sounds, electronic music composes sounds (p. 2).
\end{MyShadequote}

\subsubsection{Representation of pitch ranges}
\label{subs:koenig:pitch_ranges}

This paper deals mainly with serializm, and how it is applied to electronic music.
The following idea came from this way of thinking, in serias, but I believe that it might by understood and analyzed even without considering serial decisions.
Here, Koenig describe different ways to represent a pitch range.

\begin{MyShadequote}
  \begin{description}
    \item[Pointillistically] - single pitches are used in such a way that the limits of the range are frequently touched.

    \item[In planes] - the range is filled as continuously as possible...
    The range as a whole is clear, since limits and interior are uniformly occupied.

    \item[Articulated in planes] - the entire plane of the range is divided up into changing sections which in their turn can be either pointillist or planes:
    pointillist as ``chords'', clusters of various sizes;
    planes in the form of smaller planes within the larger one.

    \item[Transition] - it is a simple matter to have a regular transition from planes to points if the plane is brought about by a rapid sequence of points by:
    (a) slowing down the process,
    or (b) reducing the number of layers step by step (p. 11).
  \end{description}

\end{MyShadequote}

\subsection{Comparison}
\label{sub:comparison}

\begin{figure}[!htb]
  \centering
  \textbf{Categorization of ideas}\par\medskip
  \includegraphics[width=\linewidth]{graphics/concepts_categorization.pdf}
  \caption{Subjective categorization of ideas to sound, serializm and high level expressions.}
  \label{fig:concepts_categorization}
\end{figure}

Generally, I found that most of the ideas above can be divided easily into three categories --- sound, serial thinking, and higher level expressions --- as shown in Figure \ref{fig:concepts_categorization}.

The most prominent category in the composers writings is, perhaps, sound.
Both Stockhausen and Koenig count several characteristics of electronic sound which differ them from instrumental music.
Ligeti, on the other hand, composes mainly instrumental music but use very similar terminology to describe the sound he is looking for.
I think that Koenig description of non-stationary sounds in electronic music (see subsection \ref{subs:koenig:nonstationary}) may be used as a superset for all of the other concepts presented in this category.
The ``densely woven cobweb'', ``floating, fluctuating patterns'', and ``dirty smudginess'' of Ligeti are very similar to Stockhausen's ``ordering of the succession of pulses'' and Koenig's ``micro-time movement of timbre''.

It is important to note though that Ligeti's interest in the modification of timbre should be understood as part of a larger context.

\begin{MyShadequote}
  Modification of timbre and dynamics are obviously very significant but the patterns emerging from them are even more important...
  I should say that the changes and modifications of the overall pattern are the important feature, not the tone colors. (\cite[p. 39]{varnai}).
\end{MyShadequote}

The same might be applied to Stockhausen and Koenig as well, whereas the larger context is the serial way of thinking the modification of timbre is part of.
Serial aesthetics in electronic music is beyond the scope of the current work, thereby I will refrain from answering that question.

The second category I found is serializm.
As mentioned earlier, I will not deal directly with serializm in the current work.
However, there are some concepts that are rooted in serial way of thinking that can be generalized and understood by themselves.
Concepts like these, despite their problematic source, are not ignored and will be analyzed in the next section.
This category is probably the most foreign one to Ligeti's way of thinking.
Moreover, it seems that a unified model of time, as described by Stockhausen, contradict much of Ligetis' concepts from the next category.

The last category is the one for higher level expressions, and is occupied almost solely by Ligetis' ideas.
These concepts are based on expressions and feelings, and are characterized by descriptions that are not musical in their nature.

Two concept are not categorized simply into these three categories.
Noise, for example, as described by Stockhausen, can be understood in every time range;
e.g. both as a property of the sound itself, as well as of rhythm or form.
In that regard, this idea is belong both to sound and to serial thinking, as it emerged from Stockhausen unified model of time.
In the next section I will try to refer to this concepts in different time ranges.

The last concept is the representation of pitch ranges.
It is clearly a serial concept, but part of it, especially the ``transitional'' representation, reminds me the ``Mistiness'' to ``clearing-up'' process of Ligeti.
Extract this idea from its serial way of thinking, and it can be seen as a very expressive description of relationships between ``planes'' of sounds and melodies, including intermediate states.

\section{Analysis of Ligeti's Ramifications}
\label{sub:ramifications}

In this section I analyze Ligeti's Ramifications according to the concepts described before.
The order of the concepts in the analysis will follow the three categories described in the last section (\ref{sub:comparison}), starting from sound, through serializm and finally to higher level expressions.
In each category I analyze general concepts first when applicable.

The analysis was made by hearing a recording of Ramification from 2003 (\cite{rami_music}).
I found the recording more informative than the composition score;
meaning that Ligeti's ideas such as misleading notation and flexible rhythm appears more clearly in the audible result than in the score itself.
Throughout the analysis I refer to timestamps in the piece for examples based on this recording.

\subsection*{Non-stationary sounds of unlimited length}

Ligeti presents these ideas throughout the whole composition.
There are only brief sections where the music does not present an infinite texture of polyphonic lines.
The polyphonic lines, as Ligeti explains, are not meant to be heard by themselves.
Their purpose is to serve the overall texture, the ``woven cobweb''.
Koenig ideas of non-stationary sounds and sounds of unlimited length fits this composition nicely.

\subsection*{Lower level building blocks}

This concept is coupled tightly to the process of electronically creating sounds.
Hence, it is not trivial to extract some fundamental concepts from this process, to be relevant to Ramifications.
Nevertheless, Stockhausen argues that ``one would not proceed from sound properties that had already been experienced'' (\cite[p. 42]{stockhausen}).
In that regard, I think that Ligeti does not relies on already established properties of the sound and their derivatives, but rather explores different low level blocks.
Compared to the historically established melody / harmony building blocks of western music Ligetis' foundations include the already discussed concepts of micropolyphony, flexible rhythm and misleading notation.

\subsection*{Micropolyphony, flexible rhythm and misleading notation}

Beyond the general concepts above, this ideas are perhaps the most dominants in this composition.
Examples can be found in 0:00 - 1:55, 4:20 - 4:35, 5:28 - 6:25.

\subsection*{Sound as a formal event}

There are some more consonant sounds in this composition that serve somewhat as a structuring tool;
they divide the composition into sections.
Namely, the first one is found in 2:28, and it marks the end of the first part of the composition and divide it from the rest.
The second obvious sound event is found in 6:45.
It marks the beginning of the last section of the composition.

\subsection*{Noise}

Noise, as Stockhausen describe it, is found in this composition mainly in the rhythm layer, and it might be understood in two different ways:

\begin{itemize}
  \item First, the micropolyphony and flexible rhythm presented in this composition can be understood as a noise like phenomenon;
  the players repetitive melody lines are not part of a clearly structured rhythm;
  the lines might share the same beat for a moment and then gradually shift back to different tempi (1:01 - 1:24 or 5:32 - 6:25 for example).
  \item Second, the absence of audible bars can also be understood as noise.
  The bar lines are a natural part of western music, and take an important role in dividing the melody / harmony into repeatable mid-level structure.
  In Ramification they can't be heard.
  Instead, the mid-level structure is much more dynamic; one cannot anticipate it as with traditional bars.
\end{itemize}

\subsection*{The correlation of timbre, pitch, intensity and duration}

Although this idea can be understood outside of serial thinking, as I argued before, I'm unable to find its occurrence in Ramifications.
This concept is, perhaps, too contradicted to Ligeti's way of thinking.
I will refer to this issue again later, in the discussion.

\subsection*{Representation of pitch ranges}

This idea shines throughout Ramifications.
The opening of the piece serves as a good example.
From the start, Ligeti `represent' a very narrow pitch range, using a sound plane.
This sound plane is in a transitory mode most of the time;
the sound repeatedly moved back and forth between dense textures and recognizable melodic lines.
In addition, after one minute the plane starts to move and expand its higher bound to higher pitches.
This trend accelerate till 2:19, where, by slowing the patterns and reducing the layers, the plane transitions into an unison.

\subsection*{Hectic quality / Gesticulation / Deep-frozen expressionism}

Gesticulated phrases appears in Ramifications only for short periods, they are not characterize the composition as a whole.
Nevertheless, I can note two ``overly exaggerated'' phrases, where the primer is a bit more distinct for that mater: 3:36 - 3:45, 6:28 - 6:40.
One could argue that these are just dramatic phrases within the composition, but I suspect that this amount of vibrato in Ligeti's composition most represent some decent amount of sarcasm (TODO cite frigyesi).

\subsection*{Machine like music / Meccanico-type}

It appears to me that this idea could be counted together with those presented in the micropolyphony, flexible rhythm and misleading notation above.
On the other hand, some parts in Ramifications are more dominated by ``machine-like'' characteristics than others.
For example: 4:23 - 4:53, 5:28 - 6:25, 7:00 - 8:00.

\subsection*{‘Mistiness’ to ‘clearing-up’ process}

In my opinion this idea is the lessen distinct in this composition.
This is probably because of the firm voice that accompanies any of the mictrotonal parts, while the ``mistiness'' microtonal is steady and is not ``clearing-up'' into an perfect interval.
The most relevant parts I managed to find for this idea are: 1:40 - 2:08, 2:19 - 2:38.

\section{General Discussion}
\label{sec:general_discussion}

In this work I collected ideas and concepts by Ligeti, Stockhausen and Koenig, in order to show that aesthetics of electronic music could be realized in Ligeti's Ramifications, and thereby in instrumental music in general.
Even from the comparison of the composers ideas I showed that there are lots of similarities between the conceptions that Stockhausen and Koenig refer to as electronic music foundations to those of Ligeti, which are rooted in instrumental music.

Throughout this work I refrained from analyzing serial concepts per-se, and tried to extract ideas that can stand by themselves from the context of serializm when possible.
As I see it, the fact that Ligeti's music have nothing to do with serializm should not constrain the analysis of its similarity to electronically created music.
Moreover, I think that the analysis of Ramification using both electronic and instrumental concepts support this argument.

Two of Stockhausen arguments about electronic music are hard to sattle and generalize to a `way of thinking' thing.
The first is the statement that electronic music most present serializm, inherently.
A further analysis of Stockhausen's electronic music based on the concepts above may reveal more insights about this issue.
The other one, which is based on the former, is the concept of unified time model.
As already mentioned, it seems that this idea contradicts much of Ligeti's way of thinking.
Again, analysis of Stockhausen's compositions may shed more light about how this concepts are realized, or not, in electronic pieces.
Compared to Stockhausen, Koenigs' concepts apparently works great with Ligeti's music.

Generally, I think that the current work answer the research question positively, by showing that electronic music concepts are aesthetic ideals; they are more a \textbf{way of thinking} than a technique and a process of sound creation.
These concepts can be realized in both electronically created and instrumental music.
Future study might explore how this ideas translate to the music of Stockhausen and Koenig as well.
It might reveal more insights about realization of different aesthetics and how does they applied to different kinds of music.

\printbibliography[title={Bibliography}]

\end{document}
